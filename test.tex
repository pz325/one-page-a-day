\title{Zhenglin Year 6}
\author{
        Zhenglin \\
}
\date{\today}
\documentclass[12pt]{article}

\begin{document}
\maketitle

\section{3.1 Using letters to represent numbers}

\paragraph{}
\begin{itemize}
    \item $b+b+b+b\times b$ can be simply written as (~~~~~~~~)
    \item In a triangle, if $\angle 1=a^{\circ}$ and $\angle 2=b^{\circ}$, then $\angle 3=$ (~~~~~~~~).
    \item In an isosceles triangle, if the base angle is $a^{\circ}$, the degree of the vertex angle is (~~~~~~~~)    
    \item When the sum of three consecutive even number is $a$, then the number in the middle is (~~~~~~~~), the least number is (~~~~~~~) and the gratest number is (~~~~~~~~)
    \item Use expressions with letters to represent the relations between quantities.
    \begin{itemize}
        \item The quotient of $5$ divided by $x$ plus $n$ is (~~~~~~~~)
        \item $320$ is substracted by $12$ times $m$
    \end{itemize}
\end{itemize}

\section{Previous work}\label{previous work}
A much longer \LaTeXe{} example was written by Gil~\cite{Gil:02}.

\section{Results}\label{results}
In this section we describe the results.

\section{Conclusions}\label{conclusions}
We worked hard, and achieved very little.

\bibliographystyle{abbrv}
\bibliography{main}

\end{document}
This is never printed