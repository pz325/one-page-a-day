\documentclass{article}%
\usepackage[T1]{fontenc}%
\usepackage[utf8]{inputenc}%
\usepackage{lmodern}%
\usepackage{textcomp}%
\usepackage{lastpage}%
%
\title{Year 6}%
\author{Zhenglin}%
\date{\today}%
%
\begin{document}%
\normalsize%
\maketitle%
\section*{3.1 Using letters to represent numbers}%
\begin{itemize}%
\item%
$b+b+b+b\times b$ can be simply written as $\rule{2cm}{0.15mm}$%
\item%
In a triangle, if $\angle 1=a^{\circ}$ and $\angle 2=b^{\circ}$, then $\angle 3=$ $\rule{2cm}{0.15mm}$.%
\item%
Use expressions with letters to represent the relations between quantities.%
\begin{itemize}%
\item%
The quotient of $5$ divided by $x$ plus $n$ is $\rule{2cm}{0.15mm}$%
\item%
$320$ is substracted by $12$ times $m$: $\rule{2cm}{0.15mm}$%
\end{itemize}%
\item%
In an isosceles triangle, if the base angle is $a^{\circ}$, the degree of the vertex angle is $\rule{2cm}{0.15mm}$%
\item%
When the sum of three consecutive even number is $a$, then the number in the middle is $\rule{2cm}{0.15mm}$, the least number is $\rule{2cm}{0.15mm}$ and the gratest number is $\rule{2cm}{0.15mm}$%
\end{itemize}

%
\end{document}